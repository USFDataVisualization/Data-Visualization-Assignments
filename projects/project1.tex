\documentclass[a4paper,12pt]{article}

\usepackage{../my_pkg}

\title{Project 1: Visualizing Data with Tableau \\
\vspace{5pt}\normalsize (Adapted from an assignment created by James Foley)}
\SetDocumentFooter{}{}


\begin{document}

\maketitle


\section{Objectives}

\myparagraph{This assignment will familiarize you with a full-featured Information Visualization system, Tableau.
The goals of the assignment are for you to learn the capabilities provided by Tableau, practice with the basic visualization methods that it provides, apply your knowledge of visual design, and assess Tableau's utility in analyzing data.}

\vspace{5pt}
\section{Ground Rules}


\myparagraph{This assignment is intended to be done alone. You may ask others for help with figuring out how to use the program. However, the write-up and its ideas should be developed by you.}

\vspace{5pt}
\section{Assignment Instructions}

\begin{itemize}

\item Download tableau public at  \url{http://public.tableau.com}.

\item Familiarize yourself with the visualization techniques and the user interfaces via on-line videos  and tutorials at \url{http://www.tableausoftware.com/learn/training}.

\item Create a profile and find the available sample data sets (under Resources$\rightarrow$Sample Data Sets). Browse the data sets and select one to use for the rest of this assignment. You may alternatively find and select your own data set from elsewhere.
\item Once you decide upon a data set, think about interesting questions one might ask of the data set---in other words, put yourself in the shoes of a data analyst, and think about all the different kinds of analysis tasks a person might perform on your chosen data set. 
\item Develop visualizations (at least 3) to answer your questions. DON'T make all of your questions simple! Think deeply about what may be interesting in the data.
\item Place your visualizations into a Tableau Dashboard or Story, and publish it to your profile. 

\end{itemize}

\section{Submission}

\myparagraph{In the Canvas assignment, paste the URL to your published Dashboard or Story. Make sure the sharing settings are such that your submission will be visible to anyone with the URL (i.e., the instructors and classmates who will provide peer feedback).}

\newpage
\section{Grading and Feedback}

\begin{itemize}

\item Your grade will be combination of objective measures (based on the assignment instructions) and subjective grading by the instructor.

\item Breakdown

\vspace{-5pt}
\begin{itemize}
\item Dashboard/Story - 1 points
\item 3 Visualizations - 3 points
\item Professionalism - 1 points
\end{itemize}

\item Peer Review will be used to provide feedback only. You will review 3 of your peers' submissions, and 3 of your peers will review your work. This should be taken very seriously as it is the only form of detailed feedback you'll receive.

\end{itemize}






\newpage

\begin{center}
{\huge Project 1 Peer Review}
\end{center}



%https://docs.google.com/forms/d/e/1FAIpQLSeKvZ1CC6AAuGyP18EVP13sF1d7gV3QVnXqzhYIGAGngFYvEw/viewform?usp=pp_url&entry.1254628126=asdfqw


\StartTable{Visualization Narrative}

\AddElement{Choice of Visualization}
	{Was a good choice of accurate and informative visualization used?}
	{\choice No Description}
	{$\square$~Incomplete /\\$\square$~Self-Explanatory}
	{\choice Complete Description}

%\AddElement{Support of Narrative}
%	{Does the visualization support the message of the narrative?}
%	{\choice No Description}
%	{$\square$~Incomplete / $\square$~Self-Explanatory}
%	{\choice Completely Supportive}
%        
%\AddElement{Dataset Used}
%	{Do the dataset(s) provide enough information and detail to support 
%    	the narrative?}
%	{\choice Not At All}
%	{\choice Partially}
%	{\choice Completely}

\EndTable  


\vspace{15pt}
\StartTable{Design Considerations}

\AddElement{Clear, Detailed, and Thorough Labeling}
   	{Is appropriate and complete labeling used throughout or do 
      	missing labels require assumptions about the data?}
    {\choice No labels}
    {\choice Some Missing labels}
    {\choice Completely labeled} 
        
\AddElement{Missing Scales}
   	{Are scales provided for the data?}
	{\choice No Scales}
	{\choice Some Missing Scales}
	{\choice All Scales Present} 

\AddElement{Missing Legend}
	{Is a legend provided for the data? Does the legend provide useful 
    	information?}
	{\choice No Legend}
	{\choice Incomplete Legend}
	{\choice Complete Legend} 
        
        
\AddElement{Scale Distortion}
	{Is any scale distortion or deception used in the visualization?}	
	{\choice Severe Distortion}
	{\choice Minor Distortion}
	{\choice No Distortion} 

\AddElement{Lie Factor}
	{Is there any lie factor? How extreme is the lie factor?}
	{\choice Major Lie}
	{\choice Minor Lie}
	{\choice No Lie} 

\AddElement{Data/Ink Ratio}
	{Is the data to ink ratio reasonable? Could it be more efficient?}
	{Way Too $\square$~Little / $\square$~Much Ink}
	{Slightly Too $\square$~Little / $\square$~Much Ink}
	{\choice Perfect Amount of Ink} 
       
\EndTable  
        

\StartTable{Design Considerations (cont.)}

\AddElement{Chart Junk, Embellishments, Aesthetics}
	{Are appropriate embellishments used? Are the embellishments 
    	distracting? Do the embellishments add to the visualization?}
	{Way Too $\square$~Few / $\square$~Many Embellishments}
	{A Bit Too $\square$~Few / $\square$~Many Embellishments}
	{\choice Perfect Number of Embellishments} 
        
\AddElement{Data Density}
	{Has too much data been included in the visualization making 
    	interpretation difficult? } 
	{\choice Too Sparse}
	{\choice Expected}
	{\choice Too Dense} 
        
\AddElement{Task Selection}
	{Does the visualization enable appropriate visual analysis tasks for 
    	the data type and/or dataset?}
	{\choice Unclear Task Selection} 
    {\choice Some Tasks Missing} 
    {\choice Wide Range of Task Support}
        
\EndTable  
        


\end{document}
